%2multibyte Version: 5.50.0.2960 CodePage: 65001
\documentclass[journal,9pt,final,a4paper]{IEEEtran}%
\usepackage{amsfonts}
\usepackage[T1]{fontenc}
\usepackage[utf8]{inputenc}
\usepackage{amsmath}
\usepackage{amssymb}
\usepackage{graphicx}%
\setcounter{MaxMatrixCols}{30}
%TCIDATA{OutputFilter=latex2.dll}
%TCIDATA{Version=5.50.0.2960}
%TCIDATA{Codepage=65001}
%TCIDATA{LastRevised=Monday, February 05, 2024 08:45:36}
%TCIDATA{<META NAME="GraphicsSave" CONTENT="32">}
%TCIDATA{<META NAME="SaveForMode" CONTENT="1">}
%TCIDATA{BibliographyScheme=Manual}
%BeginMSIPreambleData
\providecommand{\U}[1]{\protect\rule{.1in}{.1in}}
%EndMSIPreambleData
\setlength\parindent{0pt}
\begin{document}

\title{Sygna\l y i obrazy cyfrowe}
\author{\textbf{Egzamin, 5 lutego, AD MMXXIV}\vspace*{5mm}\\Numer indeksu:
\begin{tabular}
[c]{|l|l|l|l|l|l|}\hline
&  &  &  &  & \\\hline
\end{tabular}
}
\maketitle

\begin{abstract}
Egzamin ma form\k{e} testu $n$\textbf{-krotnego wyboru} (dzisiaj $n=2$).
\newline 1) Prawid\l owa odpowied\'{z} to \texttt{'+1'}, a nieprawid\l owa (w
tym brak prawid\l owej)\ \texttt{'-1'} punkt. Uzyskanie\textbf{\ }%
\texttt{4}$\mathbf{\div}$\texttt{6} punkt\'{o}w daje ocen\k{e} \texttt{'dst'},
liczba \texttt{7}$\mathbf{\div}$\texttt{9} punkt\'{o}w oznacza oznacza
ocen\k{e} \texttt{'db'}. \newline 2) Ocena \texttt{'bdb'} zaczyna si\k{e} od
\texttt{10} punkt\'{o}w. Uzyskanie minimalnej liczby punkt\'{o}w skutkuje
ocen\k{a} \texttt{'cel'}. \newline 3) Egzamin trwa 24 minuty. \newline 4) W
trakcie egzaminu wolno korzysta\'{c} z \textbf{dowolnych NIEBIA\L KOWYCH}
zewn\k{e}trznych \'{z}r\'{o}de\l .

\end{abstract}

\begin{IEEEkeywords}
pr\'{o}bkowanie, interpolacja, aproksymacja, estymacja, entropia, kodowanie, kompresja
\end{IEEEkeywords}

\section{Pr\'{o}bkowanie}

\textbf{1. }Splataj\k{a}c funkcj\k{e} ci\k{a}g\l \k{a} $f\left(  x\right)  $ z
delt\k{a} Diraca $\delta\left(  x\right)  $%
\[
\left(  f\ast\delta\right)  \left(  x\right)  =\int_{-\infty}^{\infty}f\left(
\xi\right)  \delta\left(  \xi-x\right)  d\xi
\]
otrzymamy:

A. Warto\'{s}\'{c} (pr\'{o}bk\k{e}) funkcji $f$ w punkcie $x$.

B. Warto\'{s}\'{c} \'{s}redni\k{a} $f$ w przedziale $\left[  x-\xi
,x+\xi\right]  $.

\vspace*{0.25in}\textbf{2. }Pr\'{o}bkowanie (\emph{sampling}) dla
argument\'{o}w ca\l kowitych, $n=\ldots,-1,0,1,\ldots$, z wykorzystaniem delty
Diraca $\delta\left(  x-n\right)  $, przekszta\l ca ci\k{a}g\l \k{a}
funkcj\k{e} $f\left(  x\right)  $ w:

A. Ci\k{a}g, $\left\{  f\left(  n\right)  \right\}  $, warto\'{s}ci funkcji
dla tych argument\'{o}w.

B. Funkcj\k{e} $\left\{  f\left(  \left\lfloor x\right\rfloor \right)
-f\left(  \left\lceil x\right\rceil \right)  \right\}  $.

\section{Interpolacja}

\textbf{3. }Funkcj\k{e} $\varphi\left(  x\right)  $ nazywamy
interpoluj\k{a}c\k{a} je\'{s}li, m.in., $\varphi\left(  0\right)  =1$ oraz
$\varphi\left(  n\right)  =0$ dla $\left\vert n\right\vert =1,2,\ldots$.
Funkcj\k{a} interpoluj\k{a}c\k{a} jest\footnote{Symbol \texttt{'*'} nadal
oznacza splot.}:

A. $2\cdot\mathbf{1}_{\left[  -%
%TCIMACRO{\U{bd}}%
%BeginExpansion
\frac12
%EndExpansion
,%
%TCIMACRO{\U{bd}}%
%BeginExpansion
\frac12
%EndExpansion
\right)  }\left(  x\right)  .$

B. $\mathbf{1}_{\left[  -1,1\right)  }\left(  x\right)  \cdot\left(
1-\left\vert x\right\vert \right)  .$

\vspace*{0.25in}\textbf{4. }Aliasing przy pr\'{o}bkowaniu impulsowym (splocie
sygna\l u $f\left(  x\right)  $ z delt\k{a} Diraca $\delta\left(  x-n\right)  $)
wynika ze:

A. Zbyt ma\l ej cz\k{e}sto\'{s}ci pr\'{o}bkowania wzgl\k{e}dem widma sygna\l u.

B. Obecno\'{s}ci nieci\k{a}g\l o\'{s}ci w sygnale.

\vspace*{0.25in}\textbf{5. }Dla danego zbioru punkt\'{o}w $\left\{  n,f\left(
n\right)  \right\}  ,$ $n=-N,\ldots,N$, krzywa \l \k{a}cz\k{a}ca je kolejno
b\k{e}dzie najkr\'{o}tsza, gdy zastosujemy interpolacj\k{e} opart\k{a} o funkcj\k{e}:

A. $\mathbf{1}_{\left(  -1,1\right)  }\left(  x\right)  .$

B. $\left(  1-\left\vert x\right\vert \right)  \cdot\mathbf{1}_{\left(
-1,1\right)  }\left(  x\right)  .$

\section{Aproksymacja}

\textbf{6. }Funkcje $\varphi\left(  x\right)  $ i $\psi\left(  x\right)  $
nazywami ortogonalnymi, je\'{s}li ich iloczyn skalarny%
\[
\left\langle \varphi,\psi\right\rangle =\int_{D}\varphi\left(  x\right)
\phi\left(  x\right)  dx
\]
jest r\'{o}wny zero. Kt\'{o}re z par, dla $D=\left[  -1,1\right]  $\textbf{,}
s\k{a} ortogonalne:

A. $\varphi\left(  x\right)  =1$ i $\psi\left(  x\right)  =x.$

B. $\varphi\left(  x\right)  =\sin\left(  \pi x\right)  \mathbf{\ }$i
$\psi\left(  x\right)  =\cos\left(  \pi x\right)  .$

\vspace*{0.25in}\textbf{7. }Dla funkcji $f\left(  x\right)  $ o ograniczonej i
niezerowej energii w $D$ i dla wybranej bazy ortogonalnej $\left\{
\varphi_{m}\left(  x\right)  \right\}  ,$ $m\in0,1,\ldots\mathbf{,}$ zachodzi
twierdzenie (to\.{z}samo\'{s}\'{c}) Parsevala%
\[
\int_{D}f^{2}\left(  x\right)  dx=\sum_{m=0}^{+\infty}\alpha_{m}^{2},\text{
gdzie }\alpha=\left\langle \varphi_{m},f\right\rangle .
\]
Kt\'{o}re nier\'{o}wno\'{s}ci s\k{a} prawdziwe dla ka\.{z}dego $M$:

A. $\sum_{m=0}^{M-1}\alpha_{m}^{2}\leq\int_{D}f^{2}\left(  x\right)  dx.$

B. $\sum_{m=M}^{\infty}\alpha_{m}^{2}=\int_{D}f^{2}\left(  x\right)  dx.$

\section{Filtrowanie i estymacja}

\textbf{8. }Filtr splotowy dany jest wzorem na splot (por. zad. 1)
\[
g\left(  x\right)  =\left(  f\ast h\right)  \left(  x\right)  =\int_{-\infty
}^{\infty}f\left(  \xi-x\right)  h\left(  \xi\right)  d\xi.
\]
Je\'{s}li $f\left(  x\right)  $ i $h\left(  x\right)  $ s\k{a} ujemne, to
$g\left(  x\right)  $:

A. jest dodatnia.

B. mo\.{z}e przyjmowa\'{c} warto\'{s}ci ujemne je\'{s}li $f\left(  x\right)  $
jest nieci\k{a}g\l a.

\vspace*{0.25in}\textbf{9. }Szum \'{s}rutowy ma rozk\l ad Poissona%
\[
P\left(  X=k\left\vert \lambda\right.  \right)  =\frac{e^{-\lambda}\lambda
^{k}}{k!}.
\]
Je\'{s}li jego warto\'{s}ci oczekiwana $\lambda$ jest r\'{o}wna $1$, to w\'{o}wczas:

A. $X$ mo\.{z}e przyjmowa\'{c} ujemne.

B. Wariancja wynosi $\left\vert e^{-2\pi i}\right\vert $.

\section{Entropia}

\textbf{10. }Entropi\k{e} bezpami\k{e}ciowego \'{z}r\'{o}d\l a informacji o
$Q$ symbolach pojawiaj\k{a}cych si\k{e} z prawdopodobie\'{n}stwami $\left\{
p_{q}\right\}  $, $\sum_{q=1}^{Q}p_{q}=1,$ definiujemy jako warto\'{s}\'{c}
oczekiwan\k{a} informacji%
\[
H\left(  X\right)  =-\sum_{q=1}^{Q}p_{q}\log_{2}p_{q}.
\]
Dla \'{z}r\'{o}d\l a binarnego, tj. dla $Q=2$:

A. $H\left(  x\right)  $ jest najwi\k{e}ksza, gdy $p_{1}/p_{2}=1.$

B. $H\left(  x\right)  $ mo\.{z}e by\'{c} dowolnie bliska zeru.\vspace
*{0.25in}

\vspace*{0.25in}\textbf{11. }Niech $\left\{  c_{q}\right\}  $ oznacza kod, za
pomoc\k{a} kt\'{o}rego kodujemy symbole ze \'{z}r\'{o}d\l a o $Q$ symbolach.
\'{S}redni\k{a} d\l ugo\'{s}\'{c} kodu definiujemy jako%
\[
C\left(  X\right)  =\sum_{q=1}^{Q}p_{q}\left\vert c_{q}\right\vert ,
\]
gdzie $\left\vert c_{q}\right\vert $ to d\l ugo\'{s}\'{c} $q$-tej litery kodu.
Prawdziwe s\k{a} relacje:

A. $C\left(  X\right)  -H\left(  x\right)  \geq0.$

B. $C\left(  X\right)  -H\left(  x\right)  =1.$

\section{Kodowanie}

\vspace*{0.25in}\textbf{12. }W kodowaniu Huffmana wykorzystuje si\k{e}
rekurencyjnie spostrze\.{z}enie, \.{z}e w kodach optymalnych dwa najmniej
prawdopodobne symbole r\'{o}\.{z}ni\k{a} si\k{e} jedynie ostatnim bitem.
Kt\'{o}re z poni\.{z}szych przyk\l ad\'{o}w mog\k{a} by\'{c} kodami Huffmana
dla \'{z}r\'{o}d\l a o trzech symbolach:

A. $1,$ $01,$ $00.$

B. $0,$ $10,$ $11.$

\begin{center}
\vspace*{3in}
\end{center}

\section{Odpowiedzi\ldots}

\begin{center}%
\begin{tabular}
[c]{|l|l|l|}\hline
& A & B\\\hline\hline
\textbf{1} &  & \\\hline
2 &  & \\\hline
3 &  & \\\hline
4 &  & \\\hline
5 &  & \\\hline
\textbf{6} &  & \\\hline
7 &  & \\\hline
8 &  & \\\hline
9 &  & \\\hline
10 &  & \\\hline
\textbf{11} &  & \\\hline
12 &  & \\\hline
\end{tabular}
\vspace*{0.5in}

\textbf{Numer indeksu}%

\begin{tabular}
[c]{|l|l|l|l|l|l|}\hline
&  &  &  &  & \\\hline
\end{tabular}

\end{center}


\end{document}