%2multibyte Version: 5.50.0.2960 CodePage: 65001
\documentclass[journal,9pt,final,a4paper]{IEEEtran}%
\usepackage{amsfonts}
\usepackage[T1]{fontenc}
\usepackage[utf8]{inputenc}
\usepackage{amsmath}
\usepackage{amssymb}
\usepackage{graphicx}%
\setcounter{MaxMatrixCols}{30}
%TCIDATA{OutputFilter=latex2.dll}
%TCIDATA{Version=5.50.0.2960}
%TCIDATA{Codepage=65001}
%TCIDATA{LastRevised=Monday, February 05, 2024 08:45:36}
%TCIDATA{<META NAME="GraphicsSave" CONTENT="32">}
%TCIDATA{<META NAME="SaveForMode" CONTENT="1">}
%TCIDATA{BibliographyScheme=Manual}
%BeginMSIPreambleData
\providecommand{\U}[1]{\protect\rule{.1in}{.1in}}
%EndMSIPreambleData
\setlength\parindent{0pt}
\begin{document}

\title{Sygnały i obrazy cyfrowe}
\author{\textbf{Egzamin, 5 lutego, AD MMXXIV}\vspace*{5mm}\\Numer indeksu:
\begin{tabular}
[c]{|l|l|l|l|l|l|}\hline
&  &  &  &  & \\\hline
\end{tabular}
}
\maketitle

\begin{abstract}
Egzamin ma formę testu $n$\textbf{-krotnego wyboru} (dzisiaj $n=2$).
\newline 1) Prawidłowa odpowiedź to \texttt{'+1'}, a nieprawidłowa (w
tym brak prawidłowej)\ \texttt{'-1'} punkt. Uzyskanie\textbf{\ }%
\texttt{4}$\mathbf{\div}$\texttt{6} punktów daje ocenę \texttt{'dst'},
liczba \texttt{7}$\mathbf{\div}$\texttt{9} punktów oznacza oznacza
ocenę \texttt{'db'}. \newline 2) Ocena \texttt{'bdb'} zaczyna się od
\texttt{10} punktów. Uzyskanie minimalnej liczby punktów skutkuje
oceną \texttt{'cel'}. \newline 3) Egzamin trwa 24 minuty. \newline 4) W
trakcie egzaminu wolno korzystać z \textbf{dowolnych NIEBIAŁKOWYCH}
zewnętrznych źródeł.

\end{abstract}

\begin{IEEEkeywords}
próbkowanie, interpolacja, aproksymacja, estymacja, entropia, kodowanie, kompresja
\end{IEEEkeywords}

\section{Próbkowanie}

\textbf{1. }Splatając funkcję ciągłą $f\left(  x\right)  $ z
deltą Diraca $\delta\left(  x\right)  $%
\[
\left(  f\ast\delta\right)  \left(  x\right)  =\int_{-\infty}^{\infty}f\left(
\xi\right)  \delta\left(  \xi-x\right)  d\xi
\]
otrzymamy:

A. Wartość (próbkę) funkcji $f$ w punkcie $x$.

B. Wartość średnią $f$ w przedziale $\left[  x-\xi
,x+\xi\right]  $.

\vspace*{0.25in}\textbf{2. }Próbkowanie (\emph{sampling}) dla
argumentów całkowitych, $n=\ldots,-1,0,1,\ldots$, z wykorzystaniem delty
Diraca $\delta\left(  x-n\right)  $, przekształca ciągłą
funkcję $f\left(  x\right)  $ w:

A. Ciąg, $\left\{  f\left(  n\right)  \right\}  $, wartości funkcji
dla tych argumentów.

B. Funkcję $\left\{  f\left(  \left\lfloor x\right\rfloor \right)
-f\left(  \left\lceil x\right\rceil \right)  \right\}  $.

\section{Interpolacja}

\textbf{3. }Funkcję $\varphi\left(  x\right)  $ nazywamy
interpolującą jeśli, m.in., $\varphi\left(  0\right)  =1$ oraz
$\varphi\left(  n\right)  =0$ dla $\left\vert n\right\vert =1,2,\ldots$.
Funkcją interpolującą jest\footnote{Symbol \texttt{'*'} nadal
oznacza splot.}:

A. $2\cdot\mathbf{1}_{\left[  -%
%TCIMACRO{\U{bd}}%
%BeginExpansion
\frac12
%EndExpansion
,%
%TCIMACRO{\U{bd}}%
%BeginExpansion
\frac12
%EndExpansion
\right)  }\left(  x\right)  .$

B. $\mathbf{1}_{\left[  -1,1\right)  }\left(  x\right)  \cdot\left(
1-\left\vert x\right\vert \right)  .$

\vspace*{0.25in}\textbf{4. }Aliasing przy próbkowaniu impulsowym (splocie
sygnału $f\left(  x\right)  $ z deltą Diraca $\delta\left(  x-n\right)  $)
wynika ze:

A. Zbyt małej częstości próbkowania względem widma sygnału.

B. Obecności nieciągłości w sygnale.

\vspace*{0.25in}\textbf{5. }Dla danego zbioru punktów $\left\{  n,f\left(
n\right)  \right\}  ,$ $n=-N,\ldots,N$, krzywa łącząca je kolejno
będzie najkrótsza, gdy zastosujemy interpolację opartą o funkcję:

A. $\mathbf{1}_{\left(  -1,1\right)  }\left(  x\right)  .$

B. $\left(  1-\left\vert x\right\vert \right)  \cdot\mathbf{1}_{\left(
-1,1\right)  }\left(  x\right)  .$

\section{Aproksymacja}

\textbf{6. }Funkcje $\varphi\left(  x\right)  $ i $\psi\left(  x\right)  $
nazywami ortogonalnymi, jeśli ich iloczyn skalarny%
\[
\left\langle \varphi,\psi\right\rangle =\int_{D}\varphi\left(  x\right)
\phi\left(  x\right)  dx
\]
jest równy zero. Które z par, dla $D=\left[  -1,1\right]  $\textbf{,}
są ortogonalne:

A. $\varphi\left(  x\right)  =1$ i $\psi\left(  x\right)  =x.$

B. $\varphi\left(  x\right)  =\sin\left(  \pi x\right)  \mathbf{\ }$i
$\psi\left(  x\right)  =\cos\left(  \pi x\right)  .$

\vspace*{0.25in}\textbf{7. }Dla funkcji $f\left(  x\right)  $ o ograniczonej i
niezerowej energii w $D$ i dla wybranej bazy ortogonalnej $\left\{
\varphi_{m}\left(  x\right)  \right\}  ,$ $m\in0,1,\ldots\mathbf{,}$ zachodzi
twierdzenie (tożsamość) Parsevala%
\[
\int_{D}f^{2}\left(  x\right)  dx=\sum_{m=0}^{+\infty}\alpha_{m}^{2},\text{
gdzie }\alpha=\left\langle \varphi_{m},f\right\rangle .
\]
Które nierówności są prawdziwe dla każdego $M$:

A. $\sum_{m=0}^{M-1}\alpha_{m}^{2}\leq\int_{D}f^{2}\left(  x\right)  dx.$

B. $\sum_{m=M}^{\infty}\alpha_{m}^{2}=\int_{D}f^{2}\left(  x\right)  dx.$

\section{Filtrowanie i estymacja}

\textbf{8. }Filtr splotowy dany jest wzorem na splot (por. zad. 1)
\[
g\left(  x\right)  =\left(  f\ast h\right)  \left(  x\right)  =\int_{-\infty
}^{\infty}f\left(  \xi-x\right)  h\left(  \xi\right)  d\xi.
\]
Jeśli $f\left(  x\right)  $ i $h\left(  x\right)  $ są ujemne, to
$g\left(  x\right)  $:

A. jest dodatnia.

B. może przyjmować wartości ujemne jeśli $f\left(  x\right)  $
jest nieciągła.

\vspace*{0.25in}\textbf{9. }Szum śrutowy ma rozkład Poissona%
\[
P\left(  X=k\left\vert \lambda\right.  \right)  =\frac{e^{-\lambda}\lambda
^{k}}{k!}.
\]
Jeśli jego wartości oczekiwana $\lambda$ jest równa $1$, to wówczas:

A. $X$ może przyjmować ujemne.

B. Wariancja wynosi $\left\vert e^{-2\pi i}\right\vert $.

\section{Entropia}

\textbf{10. }Entropię bezpamięciowego źródła informacji o
$Q$ symbolach pojawiających się z prawdopodobieństwami $\left\{
p_{q}\right\}  $, $\sum_{q=1}^{Q}p_{q}=1,$ definiujemy jako wartość
oczekiwaną informacji%
\[
H\left(  X\right)  =-\sum_{q=1}^{Q}p_{q}\log_{2}p_{q}.
\]
Dla źródła binarnego, tj. dla $Q=2$:

A. $H\left(  x\right)  $ jest największa, gdy $p_{1}/p_{2}=1.$

B. $H\left(  x\right)  $ może być dowolnie bliska zeru.\vspace
*{0.25in}

\vspace*{0.25in}\textbf{11. }Niech $\left\{  c_{q}\right\}  $ oznacza kod, za
pomocą którego kodujemy symbole ze źródła o $Q$ symbolach.
Średnią długość kodu definiujemy jako%
\[
C\left(  X\right)  =\sum_{q=1}^{Q}p_{q}\left\vert c_{q}\right\vert ,
\]
gdzie $\left\vert c_{q}\right\vert $ to długość $q$-tej litery kodu.
Prawdziwe są relacje:

A. $C\left(  X\right)  -H\left(  x\right)  \geq0.$

B. $C\left(  X\right)  -H\left(  x\right)  =1.$

\section{Kodowanie}

\vspace*{0.25in}\textbf{12. }W kodowaniu Huffmana wykorzystuje się
rekurencyjnie spostrzeżenie, że w kodach optymalnych dwa najmniej
prawdopodobne symbole różnią się jedynie ostatnim bitem.
Które z poniższych przykładów mogą być kodami Huffmana
dla źródła o trzech symbolach:

A. $1,$ $01,$ $00.$

B. $0,$ $10,$ $11.$

\begin{center}
\vspace*{3in}
\end{center}

\section{Odpowiedzi\ldots}

\begin{center}%
\begin{tabular}
[c]{|l|l|l|}\hline
& A & B\\\hline\hline
\textbf{1} &  & \\\hline
2 &  & \\\hline
3 &  & \\\hline
4 &  & \\\hline
5 &  & \\\hline
\textbf{6} &  & \\\hline
7 &  & \\\hline
8 &  & \\\hline
9 &  & \\\hline
10 &  & \\\hline
\textbf{11} &  & \\\hline
12 &  & \\\hline
\end{tabular}
\vspace*{0.5in}

\textbf{Numer indeksu}%

\begin{tabular}
[c]{|l|l|l|l|l|l|}\hline
&  &  &  &  & \\\hline
\end{tabular}

\end{center}


\end{document}
